\section{Introduction}
\deleted[id=pan]{图数据在现实世界广泛存在。图数据的可视化在很多领域都有其重要作用,比如在金融数据可视分析中可视化欺诈交易,探索社交媒体网络中的信息传播,以及展示生物网络中的蛋白质互相作用等等。
为了提升用户和开发者构建可视化的效率,一系列可视化构建工具被提出。特别是d3的提出,降低了基于web的可视化构建的难度,丰富了可视化社区。因此,很多图数据可视化工具也基于web进行开发。其中,节点链接图是最为广泛使用的可视化形式。}

\added[id=pan]{Visualization of graph data plays an important role in many fields, such as showing fraud transactions in financial data analysis~\cite{DBLP:journals/tvcg/ChenGHPNXZ19}, exploring information propagation in social media network~\cite{DBLP:conf/candt/SmithSMRBDCPG09}, and visualizing protein-protein-interaction in biological network~\cite{doncheva2012topological}.
A series of visualization authoring tools have been developed to facilitate visualization generation. Notably, the proposal of d3.js~\cite{DBLP:journals/tvcg/BostockOH11} reduces the difficulty of web-based visualization authoring and enriches the visualization community. Thus, a great part of graph visualization tools are also web-based. Node-link diagrams are widely used among many graph visualization solutions, because they reveal topology and connectivities~\cite{ghoniem2004comparison}.}

% Network data wildly exists in the world. Network visualization plays an important role in many fields, such as showing fraud transactions in financial data analysis~\cite{DBLP:journals/tvcg/ChenGHPNXZ19}, exploring information propagation in social media network~\cite{DBLP:conf/candt/SmithSMRBDCPG09}, and visualizing protein-protein-interaction in biological network~\cite{doncheva2012topological}. As the development of modern browsers and open-sourced communities, a series of visualization generation tools and programming toolkits are employed to construct network visualization in a web page~\cite{DBLP:journals/tvcg/SrinivasanPEB18,DBLP:conf/ieeevast/BigelowNML19}. In the meantime, as the data grows, attention has been paid to the large-scale network visualization~\cite{DBLP:journals/tvcg/ChenGHPNXZ19}. A series of requirements and tasks requires users to explore and analyze large networks. A large-scale network visualization tool is necessary for users to rapidly and efficiently construct systems.

\deleted[id=pan]{节点链接图可视化拥有其特点,比如每一条链接往往链接了两个节点,用户只需控制节点的位置,其链接的位置会相应变化;节点链接图依赖于布局算法,从图数据到可视化的映射,需要用户通过布局算法来处理节点的位置摆放;针对节点的交互(比如点击和拖拽)非常普遍。开发者在使用通用的可视化工具进行开发时,需要pay effort to 处理这些问题。比如在使用D3.js进行图数据可视化时,开发者需要使用其数据驱动文档的思想将对应的数据映射到对应的可视化元素上,开发者直接面向可视化元素进行开发,没有抽象的模型来支持他控制整个图数据,可能会导致编程错误的产生。
一些图可视化构建工具,比如xxxx,良好地支持了对图数据的可视化,相对于通用的可视化工具而言,他们通过解决节点链接图的特殊需求,封装图可视化的相关接口,隐藏一部分开发者不关心的接口(比如对边的位置控制),暴露一部分图可视化特性接口(对相邻节点的访问),来提升自身的易用性。}

\added[id=pan]{The node-link diagrams visualization has its speciality. For example, one link in a node-link diagram connects two nodes. User only need to control the nodes' positions and related links change their positions correspondingly; node-link diagrams strongly rely on layout algorithms; interactions on nodes such as clicking and dragging are frequently needed. Developers need to pay efforts to deal with these specialities using general visualization authoring tools such as d3.js~\cite{DBLP:journals/tvcg/BostockOH11}, p5~\cite{DBLP:journals/tvcg/LiM20}, and stardust~\cite{DBLP:journals/cgf/RenLH17}. 
For example, when a developer uses d3.js to create a node-link diagram, s/he needs to map data elements to graphical marks using the data-drive-documents idea. The developer directly handles the visual elements and there is no hook/handler for s/he to handle the underlying graph model. It may lead to bugs.
Some graph visualization authoring tools such as Cytoscape.js~\cite{DBLP:journals/bioinformatics/FranzLHDSB16} and Sigma.js~\cite{DBLP:journals/jossw/Coene18} support graph visualization efficiently. Compared to general visualization tools, tools for graph visualization encapsulate related interfaces of graph visualization, hide some unrelated interfaces (e.g. controlling the position of a link) for developers, and expose some graph-related interfaces (e.g. accessing neighborhoods of a node). They improve the usability through full-fill requirements of node-link diagrams.}

\deleted[id=pan]{随着数据规模的增长,图可视化也需要处理更多graphical marks(比如节点和链接)。然而,现有大部分图可视化工具难以处理有较多graphical marks。根据我们的实验,它们在渲染较大规模数据集上存在延迟,这将会降低用户的accessibility。}

\added[id=pan]{With the growth of data scale, graph visualization authoring tools need to handle more graphical marks (e.g. nodes and links). However, most of existing tools are unable to handle a large number of graphical marks. {\color{red} According to our experiments in Section~\ref{sec:experiment}, existing graph visualization authoring tools have delays in rendering large-scale graph data (with more than ...). It reduces the user accessibility.}}

% One main bottleneck of visualizing large-scale networks is the rendering performance. Conventional tools use DOM tree, SVG, or Canvas to construct graph visualization, such as D3.js~\cite{DBLP:journals/tvcg/BostockOH11}, Cytoscape.js~\cite{DBLP:journals/bioinformatics/FranzLHDSB16}, and Echarts~\cite{DBLP:journals/vi/LiMSSZWZC18}. They can not handle a large number of elements.

% For example, the SVG performance test of D3.js indicates that rendering 2000 elements will cause a noticeable lack of smoothness (around 24 frames per second)~\cite{svg}. For optimizing the effectiveness of rendering, D3.js supports Canvas in its fifth version. Echarts and Fabric.js also use Canvas as their rendering backend. However, the Canvas performance test shows that 10,000 elements will cause an obvious lack of smoothness~\cite{canvas}.
% Now, continuous research and tools focus on rendering data by using WebGL (GPU), such as Stardust.js~\cite{DBLP:journals/cgf/RenLH17}, PixiJS~\cite{graphicslearn} and P5~\cite{DBLP:journals/tvcg/LiM20}. WebGL takes a remarkable efficiency improvement in rendering, but the challenge of WebGL programming difficulty also comes.
% Users need to master the knowledge of graphics pipeline and shader programming.
% The steer learning curve and complex API lead to the second challenge.
% It is difficult for users to construct network visualizations easily.
% Existing WebGL-based visualization generation tools have already
% encapsulated a series of API about bottom interfaces. And they aim to generate general visualization components. As a result, they are inevitable to design complex and diverse API for adapting the requirements of different visualizations.
% Complex API will hinder users who have no experience of visualization in constructing network visualization applications. Sometimes, users still need to write some shader related codes.
% In the meantime, the goal of generality needs a lot of logic code and complex data structure.
% The rendering and interaction performance will be greatly reduced in a specified visualization task due to the redundancy in design.
% It is still a difficult task for users to construct large-scale network visualization rapidly and efficiently.

\deleted[id=pan]{据我们所知,尚未存在一款工具,能够解决开发者易用性以及用户可访问性的问题。我们探索了图可视化的相关设计空间,设计并实现了NetV,一款基于web的高性能图可视化工具。其通过设计一系列图可视化相关的功能和接口来提高开发者易用性,并调用了GPU的高性能渲染能力来提高渲染效率以增加用户的可访问性。
通过和其他工具的对比实验,我们验证了NetV的accessibility和usability。
我们对该工具进行了开源以便开发者访问和贡献代码。}

\added[id=pan]{To best of our knowledge, no existing tool can address developer usability and user accessibility in the same time. We explored the design space of the node-link diagram visualization, designed and developed \name, a web-based high-performance node-link diagram visualization library. It improves developer usability through a serials of node-link diagram related functions and interfaces and increases user accessibility by utilizing the high-performance rendering ability of the GPU. 
We also evaluated the usability and accessibility of \name through several comparative experiments.
\name is now open-sourced (\url{https://netv.zjuvag.org/}) for developers to access and contribute their own code.}

% To address performance and accessibility challenges, we designed and developed \name, an open-sourced\footnote{\url{netv.zjuvag.org}}, JavaScript-based, WebGL-based library for rapidly and easily construct large-scale network visualization. \name leverages GPU processing power to render large-scale network and provide rich build-in interactions to explore networks. At the same time, \name own friendly and concise programming interfaces for developers to rapidly construct network visualization applications.