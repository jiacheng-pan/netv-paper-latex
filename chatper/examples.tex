\section{Examples}
In this section, we show diverse examples to illustrate the usage of \name.
\begin{figure*}
    \includegraphics[width=\linewidth]{fig/ex5.eps}
    \caption{
        Illustrations of (a) the customized style, (b) the label plugin, (c) the lasso selection, (d) and (e) are two large-scale datasets.
%with 74,752 nodes 261,120 edges, and 35,590 nodes 572,915 edges
    }
    \label{fig:ex5}
\end{figure*}

\subsection{Basic Graph Drawing}
\autoref{fig:ex2} shows a basic case with three parts: initialization, loading data, and rendering. The \codeword{testData} illustrates the graph data format. In this example, the initialization part hangs the canvas to the document \replaced[id=kg]{``main''}{`main'}.
\begin{figure}
    \includegraphics[width=\linewidth]{fig/ex2.eps}
    \caption{
        A basic unit of drawing three nodes and two edges.
    }
    \label{fig:ex2}
\end{figure}

\subsection{Customized Style}
The most important function of graph visualization is to draw a graph with different styles, \replaced[id=kg]{including}{such as} \added[id=kg]{the} color, stroke, radius, and position of elements.
\autoref{fig:ex5} shows the setting of customized styles by using \name. Developers can customize the style of each element and set the default style in the initialization part. In particular, \replaced[id=kg]{initialization}{initializzation} configuration items are set in the \replaced[id=kg]{``configs''}{`configs'}.


% \begin{figure}
%     \includegraphics[width=\linewidth]{fig/ex1.eps}
%     \caption{
%         Customized style.
%     }
%     \label{fig:ex1}
% \end{figure}

% \subsection{Build-in datasets}
% \name supports build-in datasets for developers to construct a graph visualization (\autoref{fig:ex3}) quickly. The build-in datasets also support the attribute and the position of each node in the graph.
% developers can the radius and the color of nodes to encode different attributes of nodes.
% \begin{figure}
%     \includegraphics[width=\linewidth]{fig/ex3.eps}
%     \caption{
%         Build-in datasets.
%     }
%     \label{fig:ex3}
% \end{figure}

\subsection{Interaction}
A series of basic interactions are supported in \name, including pan, zoom, \added[id=kg]{and} \replaced[id=kg]{mouseover}{mousueover} \deleted[id=kg]{and so no}. \autoref{fig:ex8} (b) shows several \replaced[id=kg]{built-in}{build-in} interactions.

\subsection{Plugin}

\name supports showing labels (\autoref{fig:ex5} (b)) with different drawing techniques\added[id=kg]{,} such as SVG, Canvas, and WebGL. \name also supports lasso interaction (\autoref{fig:ex5} (c)) to select nodes and different layout algorithms. \autoref{fig:ex8} (a) shows \added[id=kg]{the} plugin configurations of the label, lasso, and layout.


\begin{figure}
    \includegraphics[width=\linewidth]{fig/ex8.eps}
    \caption{
        Code examples of (a) plugin configurations, and (b) build-in interactions.
    }
    \label{fig:ex8}
\end{figure}

% \subsection{Layout}
% \name supports various graph layout algorithms. Moreover, developers can implement by combining with external layouts such as D3.js (\autoref{fig:ex4} (a)), or by using \name plugins (\autoref{fig:ex4} (b)). When combining with external layouts, \name acts as a renderer to draw graph by layout results.
% When using \name layout plugins, it supports developers to controllers of all layout stages. \autoref{fig:ex5} (b) (c) (d) show the graph visualization with force-directed layout.
% \begin{figure}
%     \includegraphics[width=\linewidth]{fig/ex4.eps}
%     \caption{
%         Layout. (a) Combining with D3.js. (b) Using \name plugins.
%     }
%     \label{fig:ex4}
% \end{figure}

\subsection{Large-\replaced[id=kg]{scale}{Scale} Graph}
\name aims to render large-scale graphs. \autoref{fig:ex5}(d) shows the graph visualization results of \replaced[id=kg]{the finan512}{Finan512} dataset~\cite{davis2011university} with 74,752 nodes and 261,129 edges. \autoref{fig:ex5}(e) shows the graph visualization results of \added[id=kg]{the} bcsstk31 dataset~\cite{davis2011university} with 35,590 nodes and 572,915 edges\replaced[id=kg]{.}{,} \deleted[id=kg]{Thanks to WebGL, }NetV.js can easily support \added[id=kg]{in} rendering millions of elements \added[id=kg]{due to its use of WebGL }.




% \begin{figure}
%     \includegraphics[width=\linewidth]{fig/ex7.eps}
%     \caption{
%         Label.
%     }
%     \label{fig:ex7}
% \end{figure}