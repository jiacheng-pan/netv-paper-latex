\documentclass[5p]{elsarticle}
% \documentclass[review]{elsarticle}
\usepackage{enumitem}
\usepackage{xspace}
\usepackage{lineno,hyperref}
\usepackage{url}
\usepackage{CJKutf8}
\usepackage{xcolor}
\usepackage{xparse}
% \usepackage[xcolor=dvipdf]{changes}
\usepackage[xcolor=dvipdf, final]{changes}
\definecolor{mygreen}{rgb}{0.17, 0.55, 0.05}
\definechangesauthor[name=JiachengPan, color=mygreen]{pan}
\setdeletedmarkup{\color{gray}{#1}}
\modulolinenumbers[5]
\hfuzz=100pt 
\hbadness=100000

\journal{Journal of \LaTeX\ Templates}

%%%%%%%%%%%%%%%%%%%%%%%
%% Elsevier bibliography styles
%%%%%%%%%%%%%%%%%%%%%%%
%% To change the style, put a % in front of the second line of the current style and
%% remove the % from the second line of the style you would like to use.
%%%%%%%%%%%%%%%%%%%%%%%

%% Numbered
%\bibliographystyle{model1-num-names}

%% Numbered without titles
%\bibliographystyle{model1a-num-names}

%% Harvard
%\bibliographystyle{model2-names.bst}\biboptions{authoryear}

%% Vancouver numbered
%\usepackage{numcompress}\bibliographystyle{model3-num-names}

%% Vancouver name/year
%\usepackage{numcompress}\bibliographystyle{model4-names}\biboptions{authoryear}

%% APA style
%\bibliographystyle{model5-names}\biboptions{authoryear}

%% AMA style
%\usepackage{numcompress}\bibliographystyle{model6-num-names}

%% `Elsevier LaTeX' style
\bibliographystyle{elsarticle-num}
%%%%%%%%%%%%%%%%%%%%%%%
\newcommand{\name}{NetV.js\xspace}

\usepackage[T1]{fontenc}
\usepackage{libertine}%% Only as example for the romans/sans fonts
\usepackage[scaled=0.85]{beramono}
\newcommand{\codeword}[1]{%
\texttt{\textbf{#1}}%
}

\begin{document}
\begin{CJK}{UTF8}{gbsn}
\begin{frontmatter}

\title{\name: A Web-based Visualization Library for Large-scale Graphs}
% \tnotetext[mytitlenote]{Fully documented templates are available in the elsarticle package on \href{http://www.ctan.org/tex-archive/macros/latex/contrib/elsarticle}{CTAN}.}

%% Group authors per affiliation:
% \author{Elsevier1,2}
% \address{Radarweg 29, Amsterdam}
% \author{Elsevier3}
% \address{Radarweg 29, Amsterdam}
%% or include affiliations in footnotes:
\author[mymainaddress,mysecondaryaddress]{Dongming Han, Jiacheng Pan, Xiaodong Zhao}
\author[mymainaddress]{Wei Chen\corref{mycorrespondingauthor}}
\cortext[mycorrespondingauthor]{Corresponding author}
\ead{chenvis@zju.edu.cn}

\address[mymainaddress]{State Key Lab of CAD\&CG, Zhejiang University, Hangzhou, Zhejiang, China}
\address[mysecondaryaddress]{Zhejiang Lab, hangzhou, zhejiang, China}

\begin{abstract}
    Graph visualization plays an important role in many fields, such as social media networks, protein-protein-interaction networks, and traffic networks. There have been many visualization design tools and programming toolkits widely used in graph-related applications. However, there remains a key challenge for high-efficient visualization of large-scale graph data. In this paper, we present \name, an open-source and WebGL-based Javascript library that supports fast visualization of large-scale graph data (up to 50 thousand nodes and 1 million edges) at an interactive frame rate with a commodity computer. Experimental results demonstrate that our library outperforms existing toolkits (Sigma.js, D3.js, Cytoscape.js, and Stardust) on both the performance and memory consumption.

\end{abstract}

\begin{keyword}
Graph\sep Graph visualization\sep Node-Link diagrams\sep Web-based visualization.
\end{keyword}

\end{frontmatter}

\linenumbers

\section{Introduction}

% \added[id=pan]{
Graph visualization plays an important role in \replaced[id=kg]{several}{many} fields, such as showing fraud transactions in financial data analysis~\cite{DBLP:journals/tvcg/ChenGHPNXZ19}, exploring information propagation in social media graph\added[id=kg]{s}~\cite{DBLP:conf/candt/SmithSMRBDCPG09}, and visualizing protein-protein-interaction in biological graph~\cite{doncheva2012topological}.
A series of visualization authoring tools~\cite{satyanarayan2014declarative,mendez2016ivolver,kim2016data} have been developed to facilitate the design and creation of graph visualizations~\cite{lu2020illustrating,lyra,lyra2}. Notably, d3.js~\cite{DBLP:journals/tvcg/BostockOH11} dramatically reduces the difficulty of web-based visualization authoring and empowers the visualization community, followed by \replaced[id=kg]{numerous}{many} web-based graph visualization tools. 
% Node-link diagrams are widely used~\cite{pan2020rcanalyzer}, because they reveal topology and connectivities~\cite{ghoniem2004comparison}.
% }

% graph data wildly exists in the world. graph visualization plays an important role in many fields, such as showing fraud transactions in financial data analysis~\cite{DBLP:journals/tvcg/ChenGHPNXZ19}, exploring information propagation in social media graph~\cite{DBLP:conf/candt/SmithSMRBDCPG09}, and visualizing protein-protein-interaction in biological graph~\cite{doncheva2012topological}. As the development of modern browsers and open-sourced communities, a series of visualization generation tools and programming toolkits are employed to construct graph visualization in a web page~\cite{DBLP:journals/tvcg/SrinivasanPEB18,DBLP:conf/ieeevast/BigelowNML19}. In the meantime, as the data grows, attention has been paid to the large-scale graph visualization~\cite{DBLP:journals/tvcg/ChenGHPNXZ19}. A series of requirements and tasks requires developers to explore and analyze large graphs. A large-scale graph visualization tool is necessary for developers to rapidly and efficiently construct systems.

% \deleted[id=pan]{
%     % 节点链接图可视化拥有其特点,比如每一条链接往往链接了两个节点,用户只需控制节点的位置,其链接的位置会相应变化;节点链接图依赖于布局算法,从图数据到可视化的映射,需要用户通过布局算法来处理节点的位置摆放;针对节点的交互(比如点击和拖拽)非常普遍。
%     开发者在使用通用的可视化工具进行开发时,需要pay effort to 掌控节点链接图的视觉元素。比如在使用D3.js进行图数据可视化时,开发者需要使用其数据驱动文档的思想将对应的数据映射到对应的可视化元素上。
%     开发者首先需要将数据实体映射到节点、链接的视觉元素,计算节点的摆放位置,并且根据链接所连接的节点来设置链接的起点位置和终点位置。开发者还需要时常更新节点链接图,比如布局改变时,或者用户拖拽节点时。
%     在可视化节点连接图时,有很多类似的需求可以被抽象和封装成为特定的功能,从而降低用户维护节点链接图的难度。

%     一些图可视化构建工具,比如xxxx,良好地支持了对图数据的可视化,相对于通用的可视化工具而言,他们通过解决节点链接图的特殊需求,封装图可视化的相关接口,隐藏一部分开发者不关心的接口(比如对边的位置控制),暴露一部分图可视化特性接口(对相邻节点的访问),来提升自身的易用性。}
% \added[id=pan]{
    Among various types of visual representations~\cite{DBLP:journals/tvcg/HermanMM00}, the node-link diagram is the most popular one~\cite{pan2020rcanalyzer}, \replaced[id=kg]{as}{because} it can reveal topology and connectivities~\cite{ghoniem2004comparison}.
    % One main bottleneck of node-link diagrams is that they strongly rely on the underlying layout algorithm. Interacting with node-link diagrams is made easy by manipulating clicking or dragging nodes. Whereas, 
    However, developers need to pay special attention on manipulating \added[id=kg]{the} visual elements of node-link diagrams using general visualization authoring tools such as \replaced[id=kg]{D3}{d3}.js~\cite{DBLP:journals/tvcg/BostockOH11}, \replaced[id=kg]{P4}{p4}~\cite{p4,DBLP:journals/tvcg/LiM20,p6}, and stardust~\cite{DBLP:journals/cgf/RenLH17}.
    For example, using \replaced[id=kg]{D3}{d3}.js to create a node-link diagram \replaced[id=kg]{requires mapping}{needs to map} data elements to graphical marks by means of the data-drive-documents scheme. 
    Developers need to map data entities (nodes and links) to visual elements (e.g., circles and lines), compute node positions, and calculate the start and the end positions of each link according to its connected nodes.
    \replaced[id=kg]{In addition, developers must}{They also need to} update the node-link diagram when the layout is changed\deleted[id=kg]{,} or \added[id=kg]{when} nodes are dragged.
    Some similar requirements of visualizing node-link diagrams can be abstracted and encapsulated into related interfaces to reduce workload.
    % However, there is no hook/handler to directly process visual elements of the underlying graph, which may easily result in errors. 
    
    Some graph visualization authoring tools such as Cytoscape.js~\cite{DBLP:journals/bioinformatics/FranzLHDSB16} and Sigma.js~\cite{DBLP:journals/jossw/Coene18}\added[id=kg]{,} provide great convenience for development by encapsulating related interfaces of graph visualization, hiding unrelated interfaces (e.g. controlling the position of a link) for developers, and exposing graph-related interfaces (e.g. accessing neighborhoods of a node). They improve the usability by leveraging the features of node-link diagrams.
% }

% \deleted[id=pan]{随着数据规模的增长,图可视化也需要处理更多graphical marks(比如节点和链接)。然而,现有大部分图可视化工具难以处理有较多graphical marks。根据我们的实验,它们在渲染较大规模数据集上存在延迟,这将会降低用户的accessibility。}

% \added[id=pan]{
For large-scale graphs, a large amount of graphical marks (e.g., nodes and edges) need to be processed. However, most tools have a limited capability in displaying graph data in real-time. As reported in Section~\ref{sec:experiment}, a heavy delay occurs in visualizing graphs whose node number is higher than \replaced[id=kg]{five}{5} thousand\deleted[id=kg]{s}, \replaced[id=kg]{thus causing}{leading} \replaced[id=kg]{unpleasant}{unpleasing} user experience\deleted[id=kg]{s}.
% }

% One main bottleneck of visualizing large-scale graphs is the rendering performance. Conventional tools use DOM tree, SVG, or Canvas to construct graph visualization, such as D3.js~\cite{DBLP:journals/tvcg/BostockOH11}, Cytoscape.js~\cite{DBLP:journals/bioinformatics/FranzLHDSB16}, and Echarts~\cite{DBLP:journals/vi/LiMSSZWZC18}. They can not handle a large number of elements.

% For example, the SVG performance test of D3.js indicates that rendering 2000 elements will cause a noticeable lack of smoothness (around 24 frames per second)~\cite{svg}. For optimizing the effectiveness of rendering, D3.js supports Canvas in its fifth version. Echarts and Fabric.js also use Canvas as their rendering backend. However, the Canvas performance test shows that 10,000 elements will cause an obvious lack of smoothness~\cite{canvas}.
% Now, continuous research and tools focus on rendering data by using WebGL (GPU), such as Stardust.js~\cite{DBLP:journals/cgf/RenLH17}, PixiJS~\cite{graphicslearn} and P5~\cite{DBLP:journals/tvcg/LiM20}. WebGL takes a remarkable efficiency improvement in rendering, but the challenge of WebGL programming difficulty also comes.
% developers need to master the knowledge of graphics pipeline and shader programming.
% The steer learning curve and complex API lead to the second challenge.
% It is difficult for developers to construct graph visualizations easily.
% Existing WebGL-based visualization generation tools have already
% encapsulated a series of API about bottom interfaces. And they aim to generate general visualization components. As a result, they are inevitable to design complex and diverse API for adapting the requirements of different visualizations.
% Complex API will hinder developers who have no experience of visualization in constructing graph visualization applications. Sometimes, developers still need to write some shader related codes.
% In the meantime, the goal of generality needs a lot of logic code and complex data structure.
% The rendering and interaction performance will be greatly reduced in a specified visualization task due to the redundancy in design.
% It is still a difficult task for developers to construct large-scale graph visualization rapidly and efficiently.

% \deleted[id=pan]{据我们所知,尚未存在一款工具,能够解决开发者易用性以及用户可访问性的问题。我们探索了图可视化的相关设计需求,设计并实现了NetV,一款基于web的高性能图可视化工具。其通过设计一系列图可视化相关的功能和接口来提高开发者易用性,并调用了GPU的高性能渲染能力来提高渲染效率以增加用户的可访问性。
% 通过和其他工具的对比实验,我们验证了NetV的accessibility和usability。
% 我们对该工具进行了开源以便开发者访问和贡献代码。}

% \added[id=pan]{
To the best of our knowledge, no existing tool can meet the requirements of both authoring usability and user accessibility at the same time. We analyze node-link diagram visualization design requirements and contribute \name (\url{https://netv.zjuvag.org/}), a web-based high-performance visualization library. It provides a high usability by means of a suite of node-link related features and interfaces and increases user accessibility by utilizing the high-performance rendering ability of GPU.
We evaluate our implementation with comparative experiments.
% We contribute \name, an open source library online (\url{https://netv.zjuvag.org/}).}

% To address performance and accessibility challenges, we designed and developed \name, an open-sourced\footnote{\url{netv.zjuvag.org}}, JavaScript-based, WebGL-based library for rapidly and easily construct large-scale graph visualization. \name leverages GPU processing power to render large-scale graph and provide rich build-in interactions to explore graphs. At the same time, \name own friendly and concise programming interfaces for developers to rapidly construct graph visualization applications.
\section{Related Work}
% \subsection{Web-based Visualization Authoring Tools}

% \subsection{Graph Visualization Authoring Tools}
Many libraries~\cite{DBLP:journals/vi/LiMSSZWZC18,mei2020datav,tableau} (tools, grammars, and frameworks) have been provided for graph-based applications~\cite{wang2018graphprotector,pan2020exemplar}.
At the beginning, developers use\added[id=kg]{d} conventional programming languages to construct graph visualization applications, such as C Sharp, C++, javaScript, and Python. Developers need to have proficient programming skills and understand the implementation mechanism~\cite{reas2003processing,reas2005processing}.
In the meantime, they spend much time in developing and debugging. To make this process easier, visualization grammars and frameworks~\cite{lyra,lyra2,heer2010declarative} allow\deleted[id=kg]{s} for detailed configurations on \added[id=kg]{the} visual channels of visualizations. Representatives include D3.js~\cite{DBLP:journals/tvcg/BostockOH11}, ECharts~\cite{DBLP:journals/vi/LiMSSZWZC18}, Vega~\cite{DBLP:journals/tvcg/SatyanarayanRHH16}, and Vega-Lite~\cite{DBLP:journals/tvcg/SatyanarayanMWH17}.


However, developers still need to \deleted[id=kg]{carefully} select complex APIs \added[id=kg]{carefully} to construct a graph\deleted[id=kg]{,} because APIs are designed for general-purpose\deleted[id=kg]{d} visualization rather than graph visualization applications. Alternatively, graph visualization tools \replaced[id=kg]{such as}{like} Cytoscape.js~\cite{DBLP:journals/bioinformatics/FranzLHDSB16}, Sigma.js~\cite{DBLP:journals/jossw/Coene18}\added[id=kg]{,} and Gephi~\cite{DBLP:conf/icwsm/BastianHJ09}\added[id=kg]{,} design and implement specific features to ease the construction process. 

% Moreover, large-scale data brings new challenges.
% Some libraries~\cite{heer2005prefuse,wickham2011ggplot2,ren2014ivisdesigner} use Canvas to render elements.
% However, these libraries can only render thousands of elements. To address this issue, PixiJS~\cite{graphicslearn}, P5~\cite{DBLP:journals/tvcg/LiM20} and Stardust~\cite{DBLP:journals/cgf/RenLH17} use GPU-based acceleration technology to render a large number of elements.
% However, these GPU-based tools are also not designed for graph visualizations.
% They still have redundancy and complex API that have nothing to do with graph visualizations. It leads to a decrease in rendering efficiency and an increase in learning costs.

\replaced[id=kg]{Meanwhile}{On the other hand}, data size poses a great challenge in terms of efficiency. Some libraries [30, 31, 32] use HTML canvas to draw visual elements, \added[id=kg]{thus }yielding \deleted[id=kg]{a} low performance. PixiJS~\cite{graphicslearn}, P5~\cite{DBLP:journals/tvcg/LiM20}\added[id=kg]{,} and Stardust~\cite{DBLP:journals/cgf/RenLH17} \replaced[id=kg]{use}{utilize} GPU-based technique\added[id=kg]{s} to make \replaced[id=kg]{themselves}{it} amenable for a large number of elements. However, they are not specifically designed for graph visualizations, and contain redundant APIs that are unnecessary for graph visualizations. \replaced[id=kg]{Nevertheless, their}{There are much potential for enhancing the} rendering efficiency\added[id=kg]{ has much potential for enhancing}. 

\name targets \replaced[id=kg]{the}{on} high-\replaced[id=kg]{efficiency}{efficient} visualization of large-scale node-link diagrams. It leverages a GPU-based visualization framework to improve the rendering performance\deleted[id=kg]{,} and design-friendly concise programming interfaces for efficient manipulation over graph elements.

% \subsection{graph Grammars and Frameworks}
% Many graph grammars and frameworks have been provided to developers for designing graph visualization applications.
% In the beginning, developers use conventional programming languages to construct graph visualization applications, such as C Sharp, C++, javaScript, and Python. Developers need to have proficient programming skills and understand the implementation mechanism.
% In the meantime, they also spend a lot of time developing and debugging. To make it easier for developers to program and develop quickly, visualization grammars and frameworks give granular control of visual channels of visualizations such as D3.js~\cite{DBLP:journals/tvcg/BostockOH11}, ECharts~\cite{DBLP:journals/vi/LiMSSZWZC18}, Vega~\cite{DBLP:journals/tvcg/SatyanarayanRHH16}, and Vega-Lite~\cite{DBLP:journals/tvcg/SatyanarayanMWH17}. Developers can use more concise tools to construct visualizations.


% However, when developers construct graph visualizations, they need to carefully select complex and different API to construct a graph, because these APIs are designed for full visualization rather than graph visualization applications. Cytoscape.js~\cite{DBLP:journals/bioinformatics/FranzLHDSB16}, sigmajs~\cite{DBLP:journals/jossw/Coene18} and Gephi~\cite{DBLP:conf/icwsm/BastianHJ09} are used to construct graph visualizations. They encapsulate a series of API which are elaborated for the graph data. Developers can use them to construct graph visualizations quickly.
% Moreover, large-scale data brings new challenges. These grammars and tools can only render thousands of elements. To address this issue, PixiJS~\cite{graphicslearn}, P5~\cite{DBLP:journals/tvcg/LiM20} and Stardust~\cite{DBLP:journals/cgf/RenLH17} used GPU-based acceleration technology to render a large number of elements.
% However, these GPU-based tools are also not designed for graph visualizations.
% They still have redundancy and complex API that have nothing to do with graph visualizations. It leads to a decrease in rendering efficiency and an increase in learning costs.

% Our \name focus on large-scale graph visualization. It uses a GPU-based rendering engine to support large-scale data and design-friendly concise programming interfaces for graph visualization construction.

% \subsection{GPU-based Visualization Rendering}





\section{Design of \name}
\begin{figure*}[htbp]
    \includegraphics[width=\linewidth]{fig/architecture.eps}
    \caption{
        \name designs: \name consists of three parts: core engine, plugins, and library interface.
    }
    \label{fig:design}
\end{figure*}

% \begin{figure}[htbp]
%     \includegraphics[width=\linewidth]{fig/xmind-02.eps}
%     \caption{
%         \name designs: \name consists of three parts: core engine, plugins, and library interface.
%     }
%     \label{fig:design}
% \end{figure}
% \begin{figure}[htbp]
%     \includegraphics[width=\linewidth]{fig/xmind-03.eps}
%     \caption{
%         \name designs: \name consists of three parts: core engine, plugins, and library interface.
%     }
%     \label{fig:design}
% \end{figure}
% \begin{figure}[htbp]
%     \includegraphics[width=\linewidth]{fig/xmind-04.eps}
%     \caption{
%         \name designs: \name consists of three parts: core engine, plugins, and library interface.
%     }
%     \label{fig:design}
% \end{figure}

\subsection {Design Requirements of \name}

\deleted[id=pan]{为了探索\name的设计空间,我们采访了3个图可视化相关的专家,调研了一系列图可视化的工具,包括 Gephi, Cytoscape.js, Sigma.js, GraphViZ,总结了如下高性能节点链接图可视化的设计需求:}

\added[id=pan] {To explore the design space of \name, we interviewed three graph visualization experts, investigated six graph visualization tools including Gephi~\cite{DBLP:conf/icwsm/BastianHJ09}, Pajek~\cite{DBLP:reference/snam/BatageljM14}, SNAP~\cite{leskovec2016snap}, Sigma.js~\cite{DBLP:journals/jossw/Coene18}, GraphViZ~\cite{Ellson03graphvizand}, and Cytoscape.js~\cite{DBLP:journals/bioinformatics/FranzLHDSB16}. We summarized the following design requirements for high-performance node-link diagram visualization: }

\begin{enumerate}
\renewcommand{\labelenumi}{\textbf{R\theenumi}}
\item \deleted[id=pan]{\textbf{需要有抽象的图模型来帮助控制图可视化}: 为了简化开发者对于可视化的操作,\name需要一个抽象图模型来控制图可视化而非直接控制图可视化的元素。该模型需要支持图的以下几点特性:}
\begin{enumerate}[R\theenumi.1]
    \item \deleted[id=pan]{\textbf{链接关联节点}: 每条链接都会关联到两个节点是图结构最基本的特性,该特性使得绘制节点链接图可以不关心链接的位置,开发者可以专注于节点的位置修改,链接会相应联动。}
    \item \deleted[id=pan]{\textbf{邻节点和邻接边的可访问性}: 我们在调研过程中发现,很多可视化系统中支持了高亮某个节点的邻接边和邻节点的交互。所有专家都一致同意为图模型增加邻节点和邻接边可访问功能的重要性。}
    \item \deleted[id=pan]{\textbf{基本图论算法的支持}: 部分算法提供了一些图论的基本算法,比如计算某些度量(如节点度数,节点centrality,图直径等),或是获取两个节点之间的最短路。专家们都一致同意图论算法对于可视化的重要性,但他们部分同意实现图论算法的必要性,其中一个专家认为该功能不属于可视化渲染库所关心的需求。}
\end{enumerate}
\item \deleted[id=pan]{\textbf{需要对拥有大量可视化元素的节点链接图提供高刷新率}: 根据我们对三位图可视化专家的采访,他们认为【对超过10万元素的大规模图有30fps以上的渲染速度】才能认为其对大规模节点链接图提供了高性能渲染的能力。}
\item \deleted[id=pan]{\textbf{支持不同节点和链接的样式}: 开发者往往需要在节点链接图上编码不同信息,虽然在大部分大规模图可视化案例中,圆形的节点和直线边最受欢迎,但仍然有很多节点链接图支持了不同形状的节点和不同样式的边。因此,为节点和链接赋予不同的形状和样式,能使得开发者有更多可以编码信息的空间。}
\item \deleted[id=pan]{\textbf{需要提供多种布局功能和自定义布局插件}:节点链接图对于布局的依赖程度不言而喻。\name 需要提供一些基础的节点链接图布局。因为布局算法的多样性,\name 还需能使开发者按照既定接口接入自定义布局。}
\item \deleted[id=pan]{\textbf{需要提供自定义标签渲染}:虽然在渲染大规模节点链接图时,为可视化元素赋予标签会导致visual clutter的问题,但开发者仍有可能采取某些策略来绘制标签,比如在屏幕内元素数量较少时自适应地绘制标签。所以,\name有必要提供标签绘制的接口。}
\item \deleted[id=pan]{\textbf{需要提供基础交互模型}:虽然对于节点链接图的交互多种多样,但根据我们的调研和专家访谈,我们发现它们基本上都可以被分解为对节点、链接以及画布的交互监听。我们认为\name需要能够对节点进行拖拽、点击、鼠标悬浮的监听,对链接需要有点击和鼠标悬浮的监听,以及对画布需要能够有平移、缩放、点击的交互监听。我们还需要实现对节点链接图的可视化元素的选择功能,比如lasso。}
\end{enumerate}

\subsection{Design Details of \name}
\newcommand{\RenEng}{\textit{Rendering Engine}}
\newcommand{\GraModMan}{\textit{Graph Model Manager}}
\newcommand{\IntMan}{\textit{Interaction Manager}}

\deleted[id=pan]{为了完成上述的需求,我们设计实现了 \name。\name 包含了三个主要部分:\GraModMan 、\RenEng 以及 \IntMan 。我们还将一些非核心的需求独立抽象为插件的形式方便开发者自定义调用,从而减少\name的核心代码。}

\subsubsection{\GraModMan}
为了能够有抽象的图模型来帮助图可视化,我们在\name中设计实现了\GraModMan,一个图模型的控制器(图1)。该控制器的主体是一个Data Container,其存储了节点数据和链接数据。每个节点拥有一个唯一标志符id,其对应的样式会被存储在style属性中,而位置坐标责备存储在position中,其余的属性则会被存储在attributes中;同样的,每个link都拥有一个source和一个target,代表它所关联的两个节点,其对应的样式和其余属性会被存储在style和attributes中。我们为节点和链接建立了对应的class:Node和Link来存储每一个数据实体。他们派生自Element类。

我们为data elements (nodes和links) 设计了 增、删、改、查和计算的功能。开发者可以通过data setter来增加和删除elements,通过 data getter 则能用于查询element, 通过 data modifier 修改数据的内容(如属性、样式等)。除此以外,我们还提供了部分常用的图计算的接口,因为该部分和可视化的关联并不紧密,我们只实现了部分功能,比如查询节点的邻节点等。

\subsubsection{\RenEng}
为了满足\textbf{R2},\RenEng 调用了GPU的高性能渲染来绘制可视化。
\subsubsection{\IntMan}

% % \name aims to help users rapidly and efficiently construct network visualization applications.
% % Specifically, \name consists of three parts: the core engine, plugins, and library interface (\autoref{fig:design}).
% % The core engine contains the data manager for maintaining nodes and links and the renderer for leveraging GPU processing power to render a large-scale network.
% % The plugins are employed to increase and expand more requirements and functions.
% % The library interface aims to help developers rapidly construct applications with friendly and concise APIs.


% % \subsection{Core Engine}
% % The core engine aims to render large-scale network based on WebGL and maintains the primary network information for rapidly editing operations.
% % It consists of the data manager and the renderer.

% % \subsubsection{Data Manager}
% % The data manager supports a series of interface corresponding network structure.
% % It is used to achieve nodes or links operations, including adding, deleting, searching, and editing.
% % Node-links and link-nodes mapping tables are also supported for accelerating search and locate.
% % At the same time, the data manager has a build-in network data set for developers to get started quickly.

% % \subsubsection{Renderer}
% % The renderer aims to render basic elements: nodes and links. It focuses on the efficiency of rendering massive data on the browser platform. As the highest performance graphics rendering API of browser platform, \name uses WebGL as the bottom rendering.
% % However, WebGL programming is still hard and complex.
% % The WebGL API is encapsulate
% % In order to reduce the program execution time as much as possible, the basic WebGL API is only encapsulated to meet the most basic data processing and rendering.
% % Specifically, the renderer uses three strategies to improve rendering efficiency.

% \begin{itemize}
% \item \textbf{Batch}: The renderer contains a batch drawing element instance. Considering that most of the elements of the network are the same, but the location information is different. The renderer creates an instance of an element and draws it in the batch process to reduce the consuming time of the rendering process.

% \item \textbf{Shader}: The renderer uses Shader function to control the shape render. Each node's shape usually needs to be defined as a circle, square, ellipse, and so on. However, the complex shape will cause a serious effect on rendering performance.
% \name exploits the powerful Shader function to define and render shape in GPU with batch processing.

% \item \textbf{Modify as needed}: The renderer supports manual rendering function for developers.
% The main attributes of elements, including positions, color, and texture, are stored into different buffers of GPU.
% When attributes of elements need to be Modified, the renderer can refresh corresponding buffers rather getting attributes from the data manager. And then, developers can refresh the network by supported function. The time consuming of getting attributes from the data manager can be omitted.

% \item \textbf{Element positioning}: The renderer has an elements positioning function for supporting elements search and interaction. For improving user interaction efficiency, the renderer uses the WebGL Texture to record the screen pixel position of each element.
% The time consuming of elements positioning has great improvement compared with index search and spatial index tree.
% The time complexity of the function is $O(1)$.

% \end{itemize}

% \subsection{Plugin module}
% The goal of the plugin modular is to enhance future expansions.
% For supporting more features and requirements, \name designs the plugin modular to employ new improvements or functions.
% At the same time, it can isolate the core modular from the rest.
% For now, the plugin module includes:
% \begin{itemize}
%     \item \textbf{Interaction: } It supports binding interaction events and callback functions of elements. Interaction events contain basis events of edges and nodes such as Hover, MouseDown, Click, etc.
%     \item \textbf{Layout: } It contains build-int layouts and allows users to use custom layouts.

% \end{itemize}

% \subsection{Library Interface}
% The library interface aims to support concise and efficient API for users to build large-scale network visual analysis applications quickly.
% Users do not need to touch the underlying WebGL rendering programming. Humanized and simple API can be called to config data and render network.
% Specifically, it includes three parts:
% \begin{itemize}
%     \item \textbf{Global: }It is used to define custom configs, such as the mount node, the default style of the canvas, and the default style of elements.
%     \item  \textbf{Element: }It includes the data adding of nodes of edges and the attributes changing of elements.
%     \item  \textbf{Plugin: }It aims to set up different plugins, such as layouts and interactions.
% \end{itemize}

\section{Examples}
In this section, we show diverse examples to illustrate the usage of \name.
\begin{figure*}
    \includegraphics[width=\linewidth]{fig/ex5.eps}
    \caption{
        Illustrations of (a) the customized style, (b) the label plugin, (c) the lasso selection, (d) and (e) are two large-scale datasets.
%with 74,752 nodes 261,120 edges, and 35,590 nodes 572,915 edges
    }
    \label{fig:ex5}
\end{figure*}

\subsection{Basic Graph Drawing}
\autoref{fig:ex2} shows a basic case with three parts: initialization, loading data, and rendering. The \codeword{testData} illustrates the graph data format. In this example, the initialization part hangs the canvas to the document \replaced[id=kg]{``main''}{`main'}.
\begin{figure}
    \includegraphics[width=\linewidth]{fig/ex2.eps}
    \caption{
        A basic unit of drawing three nodes and two edges.
    }
    \label{fig:ex2}
\end{figure}

\subsection{Customized Style}
The most important function of graph visualization is to draw a graph with different styles, \replaced[id=kg]{including}{such as} \added[id=kg]{the} color, stroke, radius, and position of elements.
\autoref{fig:ex5} shows the setting of customized styles by using \name. Developers can customize the style of each element and set the default style in the initialization part. In particular, \replaced[id=kg]{initialization}{initializzation} configuration items are set in the \replaced[id=kg]{``configs''}{`configs'}.


% \begin{figure}
%     \includegraphics[width=\linewidth]{fig/ex1.eps}
%     \caption{
%         Customized style.
%     }
%     \label{fig:ex1}
% \end{figure}

% \subsection{Build-in datasets}
% \name supports build-in datasets for developers to construct a graph visualization (\autoref{fig:ex3}) quickly. The build-in datasets also support the attribute and the position of each node in the graph.
% developers can the radius and the color of nodes to encode different attributes of nodes.
% \begin{figure}
%     \includegraphics[width=\linewidth]{fig/ex3.eps}
%     \caption{
%         Build-in datasets.
%     }
%     \label{fig:ex3}
% \end{figure}

\subsection{Interaction}
A series of basic interactions are supported in \name, including pan, zoom, \added[id=kg]{and} \replaced[id=kg]{mouseover}{mousueover} \deleted[id=kg]{and so no}. \autoref{fig:ex8} (b) shows several \replaced[id=kg]{built-in}{build-in} interactions.

\subsection{Plugin}

\name supports showing labels (\autoref{fig:ex5} (b)) with different drawing techniques\added[id=kg]{,} such as SVG, Canvas, and WebGL. \name also supports lasso interaction (\autoref{fig:ex5} (c)) to select nodes and different layout algorithms. \autoref{fig:ex8} (a) shows \added[id=kg]{the} plugin configurations of the label, lasso, and layout.


\begin{figure}
    \includegraphics[width=\linewidth]{fig/ex8.eps}
    \caption{
        Code examples of (a) plugin configurations, and (b) build-in interactions.
    }
    \label{fig:ex8}
\end{figure}

% \subsection{Layout}
% \name supports various graph layout algorithms. Moreover, developers can implement by combining with external layouts such as D3.js (\autoref{fig:ex4} (a)), or by using \name plugins (\autoref{fig:ex4} (b)). When combining with external layouts, \name acts as a renderer to draw graph by layout results.
% When using \name layout plugins, it supports developers to controllers of all layout stages. \autoref{fig:ex5} (b) (c) (d) show the graph visualization with force-directed layout.
% \begin{figure}
%     \includegraphics[width=\linewidth]{fig/ex4.eps}
%     \caption{
%         Layout. (a) Combining with D3.js. (b) Using \name plugins.
%     }
%     \label{fig:ex4}
% \end{figure}

\subsection{Large-\replaced[id=kg]{scale}{Scale} Graph}
\name aims to render large-scale graphs. \autoref{fig:ex5}(d) shows the graph visualization results of \replaced[id=kg]{the finan512}{Finan512} dataset~\cite{davis2011university} with 74,752 nodes and 261,129 edges. \autoref{fig:ex5}(e) shows the graph visualization results of \added[id=kg]{the} bcsstk31 dataset~\cite{davis2011university} with 35,590 nodes and 572,915 edges\replaced[id=kg]{.}{,} \deleted[id=kg]{Thanks to WebGL, }NetV.js can easily support \added[id=kg]{in} rendering millions of elements \added[id=kg]{due to its use of WebGL }.




% \begin{figure}
%     \includegraphics[width=\linewidth]{fig/ex7.eps}
%     \caption{
%         Label.
%     }
%     \label{fig:ex7}
% \end{figure}
\section{Experiment}\label{sec:experiment}
\autoref{fig:eva} shows the \replaced[id=kg]{FPS}{frame per second (FPS)} over datasets with varied sizes of \name and other \replaced[id=kg]{five}{5} tools\footnote{https://github.com/ZJUVAG/NetV.js/tree/benchmarks/benchmarks}, namely,  % We recorded the frame per second (FPS) experiment\footnote{https://github.com/ZJUVAG/NetV.js/tree/benchmarks/benchmarks} to test the rendering performance of \name with other popular tools and libraries which are supports graph rendering, including 
D3-SVG, D3-Canvas, Cytoscape.js, Sigma.js, and Stardust.js. In particular, \name, Sigma.js, and Stardust.js use WebGL. To simulate real-world graph data, we set the graph density as 20, which means the ratio of the number of edges to the number of nodes is 1 to 20. 
% The display refresh rate is 144Hz; the GPU is GTX 1060 with 6G.
Experiments are performed on a PC equipped with a GPU (NIVIDIA GTX 1060, 6G) and a monitor with a refresh\deleted[id=kg]{ing} rate of 144 HZ.

Stardust.js, and D3-Canvas can render up to 100 thousand elements. \added[id=kg]{Meanwhile, }\name can render more than 1 million elements with FPS greater than 1.

\begin{figure}[htbp]
    \includegraphics[width=\linewidth]{fig/eva.eps}
    \caption{
        Performance comparison among \replaced[id=kg]{six}{6} toolkits.
    }
    \label{fig:eva}
\end{figure}

\section{Conclusion}
This paper presents NetV.js, an open-sourced, WebGL-based Javascript library for large-scale node-link diagrams. In the future, we plan to extend \name to support heterogeneous graphs. We also plan to explore more visualization components and visual analysis algorithms for analyzing graph data.

% \section{The Elsevier article class}

% \paragraph{Installation} If the document class \emph{elsarticle} is not available on your computer, you can download and install the system package \emph{texlive-publishers} (Linux) or install the \LaTeX\ package \emph{elsarticle} using the package manager of your \TeX\ installation, which is typically \TeX\ Live or Mik\TeX.

% \paragraph{Usage} Once the package is properly installed, you can use the document class \emph{elsarticle} to create a manuscript. Please make sure that your manuscript follows the guidelines in the Guide for Authors of the relevant journal. It is not necessary to typeset your manuscript in exactly the same way as an article, unless you are submitting to a camera-ready copy (CRC) journal.

% \paragraph{Functionality} The Elsevier article class is based on the standard article class and supports almost all of the functionality of that class. In addition, it features commands and options to format the
% \begin{itemize}
% \item document style
% \item baselineskip
% \item front matter
% \item keywords and MSC codes
% \item theorems, definitions and proofs
% \item lables of enumerations
% \item citation style and labeling.
% \end{itemize}

% \section{Front matter}

% The author names and affiliations could be formatted in two ways:
% \begin{enumerate}[(1)]
% \item Group the authors per affiliation.
% \item Use footnotes to indicate the affiliations.
% \end{enumerate}
% See the front matter of this document for examples. You are recommended to conform your choice to the journal you are submitting to.

% \section{Bibliography styles}

% There are various bibliography styles available. You can select the style of your choice in the preamble of this document. These styles are Elsevier styles based on standard styles like Harvard and Vancouver. Please use Bib\TeX\ to generate your bibliography and include DOIs whenever available.

% Here are two sample references: \cite{Feynman1963118,Dirac1953888}.

% \section*{References}

\bibliography{mybibfile}
\end{CJK}
\end{document}